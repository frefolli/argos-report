\documentclass[a4paper,11pt,oneside, table]{article}
\usepackage[margin=1in]{geometry}
\usepackage{setspace}
\usepackage{imakeidx}
\usepackage{float}
\usepackage{graphicx}
\usepackage{pdfpages}
\usepackage{csquotes}
\usepackage{caption}
\captionsetup[table]{labelfont=it}
\usepackage{pifont}% http://ctan.org/pkg/pifont

\newcommand{\cmark}{\ding{51}}%
\newcommand{\xmark}{\ding{55}}%

\usepackage{listings}
\usepackage{listings-cpp}
\usepackage{algorithm}
\usepackage{algpseudocode}

\newtheorem{nota}{Nota}

\usepackage[italian]{babel}
\usepackage[
  backend=bibtex,
  style=numeric,
  sorting=ydnt
  ]{biblatex}
\addbibresource{refs.bib}
\makeindex

\newcommand{\putimage}[4] {
	\begin{figure}[H]
	    \centering
	    \includegraphics[width={#4}\linewidth]{#1}
	    \caption{#2}\label{#3}
	\end{figure}
}

\newcommand{\putsubimage}[5] {
  \begin{minipage}{{#4}\linewidth}
	    \centering
      \includegraphics[width={#5}\linewidth]{#1}
	    \caption{#2}\label{#3}
	\end{minipage}
}

\newcommand{\putimagecouple}[2] {
  \begin{figure}[!htb]
      \centering
      #1
      \hspace{0.5cm}
      #2
  \end{figure}
}

\newcommand{\putimagequadruple}[4] {
  \begin{figure}[!htb]
      \centering
      #1
      \hspace{0.5cm}
      #2
      \linebreak
      #3
      \hspace{0.5cm}
      #4
  \end{figure}
}

\begin{document}
    \begin{titlepage}
        \noindent
        \begin{minipage}[t]{0.19\textwidth}
            \vspace{-4mm}{\includegraphics[scale=1.15]{logo_unimib.pdf}}
        \end{minipage}
        \begin{minipage}[t]{0.81\textwidth}
        {
                \setstretch{1.42}
                {\textsc{Università degli Studi di Milano - Bicocca}} \\
                \textbf{Scuola di Scienze} \\
                \textbf{Dipartimento di Informatica, Sistemistica e Comunicazione} \\
                \textbf{Corso di laurea magistrale in Informatica} \\
                \par
        }
        \end{minipage}
    	\vspace{40mm}
    	\begin{center}
            {\LARGE{
                    \setstretch{1.2}
                    \textbf{Relazione di Sistemi Complessi: Modelli e Simulazioni}
                    \par
            }}
        \end{center}
        
        \vspace{50mm}
        
        \vspace{15mm}

        \begin{flushright}
            {\large \textbf{Relazione di:}} \\
            \large{Preziosa Alessandro} \large{866142} \\
            \large{Refolli Francesco} \large{865955}
        \end{flushright}
        
        \vspace{40mm}
        \begin{center}
            {\large{\bf Anno Accademico 2023-2024}}
        \end{center}
        \restoregeometry
    \end{titlepage}

    \printindex
    \tableofcontents
    \renewcommand{\baselinestretch}{1.5}

\printbibliography[title={Bibliografia}]

\section{Introduzione}

La \textbf{Swarm Robotics} \`e un filone di ricerca della robotica che si occupa di costruire sistemi intelligenti formati da una moltitudine di droni/robot indipendenti (eventualmente eterogenei) per raggiungere tramite algoritmi fortemente decentralizzati un obiettivo di collettivo interesse (rispetto allo sciame).
Le applicazioni della Swarm Robotics spaziano in campo civile e militare.
Solitamente la ricerca si concentra su scenari che prevedono il salvataggio, la ricognizione o una battaglia.
Tuttavia essendo un concetto molto astratto vengono anche applicati alla costruzione di algoritmi di ottimizzazione.

Per la realizzazione di questi esperimenti sono stati costruiti diversi simulatori e qualche framework fisico per l'implementazione di sciami di robot dal basso costo.
L'obiettivo del progetto \`e simulare uno sciame il cui scopo \`e dividersi in squadroni, decollare ed approcciare degli obiettivi sparsi nell'ambiente.
Per la simulazione ci siamo avvalsi di ARGoS, un simulatore programmabile Free and Open Source.

\subsection{ARGoS}

ARGoS \`e un simulatore di droni che permette di utilizzare diversi engine fisici a 2 o 3 dimensioni e aggiungere programmaticamente plugin per nuovi droni o nuovi contenuti.

Per comandare un drone si deve implementare un \textbf{Controller}, il quale tramite \textbf{Attuatori} e \textbf{Sensori} pu\`o interagine con l'ambiente e con gli altri droni.
Si possono inoltre costruire delle \textbf{Loop Function}, funzioni solitamente utilizzate per interagire con la simulazione, e \textbf{User Functions}, funzioni che permettono di interagire con lo stack grafico di ARGoS.
Queste integrazioni vengono realizzate sotto forma di plugins che possono essere caricati dinamicamente tramite una configurazione XML di una simulazione.

\putimage{images/architecture.png}{Architettura di ARGoS}{png:architecture}

\subsection{Modello}

\putimage{images/simulation-model.png}{Modello della Simulazione}{png:simulation-model}

\section{Strategie implementate}

\section{Risultati}

\section{Conclusione}

\end{document}
