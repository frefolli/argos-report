\documentclass{beamer}
\usepackage{graphicx}

\usetheme{Madrid}
\usecolortheme{default}

\title{Sistemi Complessi: Modelli e Simulazioni}
\subtitle{Sciami di Droni con ARGoS}
\author{Preziosa~A.~866142 \and ~Refolli~F.~865955}
%\logo{\includegraphics[height=1cm]{logo_unimib.pdf}}

\newcommand{\putimage}[2] {
  \begin{figure}[H]
    \centering
    \includegraphics[width=#2\linewidth]{#1}
	\end{figure}
}

\newcommand{\putimagecouple}[4] {
  \begin{figure}[!htb]
    \centering
    \begin{minipage}{0.45\linewidth}
      \centering
      \includegraphics[width=\linewidth]{#1}
      \caption{#2}
    \end{minipage}
    \hspace{0.25cm}
    \begin{minipage}{0.45\linewidth}
      \centering
      \includegraphics[width=\linewidth]{#3}
      \caption{#4}
    \end{minipage}
  \end{figure}
}

\begin{document}

\frame{\titlepage}

\begin{frame}
\frametitle{Indice}
\tableofcontents
\end{frame}

\section{Introduzione}
\begin{frame}
\centering
\Huge
Introduzione
\end{frame}

\begin{frame}
\frametitle{Swarm Robotics}
  \putimage{images/swarm\_robotics.png}{0.99}
\end{frame}

\begin{frame}
\frametitle{ARGoS}
\putimage{images/argos3.png}{0.85}
\end{frame}

\begin{frame}
\frametitle{Architettura di ARGoS}
\putimage{images/architecture.png}{0.85}
\end{frame}

\section{La Simulazione}
\begin{frame}
\centering
\Huge
La Simulazione
\end{frame}

\begin{frame}
\frametitle{Modello della Simulazione}
\putimage{images/simulation-model.png}{0.99}
\end{frame}

\begin{frame}
\frametitle{Task Allocator}
\putimage{images/task-allocator.png}{0.70}
\end{frame}

\begin{frame}
\frametitle{Varianti del Task Allocator}
Nelle prove sul task allocator variano i seguenti componenti:

\begin{itemize}
  \item Scelta iniziale target: \textbf{Random} o \textbf{Nearest}.
  \item \textbf{Review}: presente o no.
  \item (se review presente) Review: Probable \textbf{Minority} vs Probable \textbf{Random}
\end{itemize}

Durante le prove sul task allocator veniva usata la variante di task executor con decollo verticale, LP come potenziale e senza rumore negli attuatori.
\end{frame}

\begin{frame}
\frametitle{Task Executor}
\putimage{images/task-executor.png}{0.99}
\end{frame}

\begin{frame}
\frametitle{Task Executor: I Potenziali}
Detta $d$ la distanza tra due corpi soggetti alla forza repulsiva (i droni), $A$ un moltiplicatore specifico di ogni potenziale utilizzato per ottimizzarne l'intensit\`a e $D$ una distanza media che si vuole mantenere tra due droni, si riportano le formule per ricavare le forze di attrazione:

\begin{itemize}
  \item $GP(d) = -A_{GP} \frac {|D - d|} {d}$
  \item $JP(d) = -A_{JP} \frac {D - d} {d^2}$
  \item $LP(d) = -A_{LP} 4 ({\frac {D} {d}}^6 - {\frac {D} {d}}^{12})$
\end{itemize}

Si riportano anche i valori dei coefficienti $A$ che abbiamo utilizzato:

\begin{itemize}
  \item $A_{GP} = 4.0$
  \item $A_{JP} = 16.0$
  \item $A_{LP} = 0.2$
\end{itemize}
\end{frame}

\begin{frame}
\frametitle{Varianti del Task Executor}
Nelle prove sul task executor variano i seguenti componenti:

\begin{itemize}
  \item \textbf{Potenziale}: uno tra LP, GP e JP.
  \item \textbf{Rumore} nell'attuatore: presente o no.
  \item Tipo di decollo: \textbf{Verticale} o \textbf{Diretto}.
\end{itemize}

Durante le prove sul task executor veniva usata la variante di task allocator con ???
\end{frame}

\section{Esempio}
\begin{frame}
\centering
\Huge
Esempio
\end{frame}

\begin{frame}
\frametitle{Situazione Iniziale}
\putimage{images/esempio/iterazione\_zero\_left.png}{0.85}
\end{frame}

\begin{frame}
\frametitle{Fase di Ascesa}
\putimage{images/esempio/iterazione\_venti\_left.png}{0.85}
\end{frame}

\begin{frame}
\frametitle{Situazione Finale}
\putimage{images/esempio/iterazione\_finale\_top.png}{0.85}
\end{frame}

\section{Esperimenti}
\begin{frame}
\centering
\Huge
Esperimenti
\end{frame}

\subsection{Task Executor}

\begin{frame}
\frametitle{I Potenziali a Confronto / 1}
  \putimage{images/slides/task-executor-noiseless-vertical/MeanDistanceFromTarget.png}{0.66}
\end{frame}

\begin{frame}
\frametitle{I Potenziali a Confronto / 2}
\putimage{images/slides/task-executor-noiseless-vertical/MeanDistancesWithinSquadron.png}{0.66}
\end{frame}

\begin{frame}
\frametitle{I Potenziali a Confronto / 3}
\putimage{images/slides/task-executor-noiseless-vertical/MinDistancesGlobally.png}{0.66}
\end{frame}

\begin{frame}
\frametitle{Decollo Verticale e Decollo Diretto / 1}
\putimagecouple{images/slides/task-executor-noiseless-vertical/MeanSpeed.png}{Velocit\`a media voluta dai droni nel tempo con Decollo Verticale}
               {images/slides/task-executor-noiseless-direct/MeanSpeed.png}{Velocit\`a media voluta dai droni nel tempo con Decollo Diretto}
\end{frame}

\begin{frame}
\frametitle{Decollo Verticale e Decollo Diretto / 2}
\putimagecouple{images/slides/task-executor-noiseless-vertical/MinDistancesGlobally.png}{Distanza minima tra droni nel tempo con Decollo Verticale}
               {images/slides/task-executor-noiseless-direct/MinDistancesGlobally.png}{Distanza minima tra droni nel tempo con Decollo Diretto}
\end{frame}

\begin{frame}
\frametitle{Aggiunta di Rumore}
\putimagecouple{images/slides/task-executor-noisy-vertical/MeanSpeed.png}{Velocit\`a media voluta dai droni nel tempo con Decollo Verticale e Rumore}
               {images/slides/task-executor-noisy-direct/MeanSpeed.png}{Velocit\`a media voluta dai droni nel tempo con Decollo Diretto e Rumore}
\end{frame}

\subsection{Task Allocator}

\begin{frame}
\frametitle{Scelta Iniziale: Random vs Nearest / 1}
\putimagecouple{images/slides/task-allocator-nearest/MeanDistanceFromTarget.png}{Distanza media dal target nel tempo con scelta iniziale Nearest}
               {images/slides/task-allocator-random/MeanDistanceFromTarget.png}{Distanza media dal target nel tempo con scelta iniziale Random}
\end{frame}

\begin{frame}
\frametitle{Scelta Iniziale: Random vs Nearest / 2}
\putimage{images/slides/task-allocator-no-review/MeanDistancesWithinSquadron.png}{0.66}
\end{frame}

\begin{frame}
\frametitle{Fase di Review / 1}
\putimagecouple{images/slides/task-allocator-nearest-shortened/MeanDistanceFromTarget.png}{distanza media dei droni dai target con scelta iniziale Nearest}
               {images/slides/task-allocator-random-shortened/MeanDistanceFromTarget.png}{distanza media dei droni dai target con scelta iniziale Random}
\end{frame}

\begin{frame}
\frametitle{Fase di Review / 2}
\putimagecouple{images/slides/task-allocator-random-shortened/VarTargetDensityOverTime.png}{varianza di distribuzione dei droni sui target con probabilit\`a di cambio bassa}
               {images/slides/task-allocator-random-always/VarTargetDensityOverTime.png}{confronto varianza di distribuzione dei droni sui target con probabilit\`a di cambio bassa e alta}
\end{frame}

\section{I Limiti}

\begin{frame}
\frametitle{I Limiti}
\centering
\Huge
ARGoS
\end{frame}

\begin{frame}
\frametitle{I Limiti}
\centering
\Huge
Forze di Separazione
\end{frame}

\begin{frame}
\frametitle{I Limiti}
\centering
\Huge
Modellazione delle Collisioni
\end{frame}

\begin{frame}
\frametitle{I Limiti}
\centering
\Huge
Criterio di Arresto
\end{frame}

\begin{frame}
\frametitle{I Limiti}
\centering
\Huge
Generalit\`a
\end{frame}

\section{Conclusioni}

\begin{frame}
\centering
\Huge
Conclusioni
\end{frame}

\begin{frame}
\centering
\Huge
Fine
\end{frame}

\end{document}
