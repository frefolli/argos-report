\documentclass{beamer}
\usepackage{graphicx}

\usetheme{Madrid}
\usecolortheme{default}

\title{Sistemi Complessi: Modelli e Simulazioni}
\subtitle{Sciami di Droni con ARGoS}
\author{Preziosa~A.~866142 \and ~Refolli~F.~865955}
%\logo{\includegraphics[height=1cm]{logo_unimib.pdf}}

\newcommand{\putimage}[2] {
  \begin{figure}
    \centering
    \includegraphics[width=#2\linewidth]{#1}
	\end{figure}
}

\newcommand{\putimagecouple}[4] {
  \begin{figure}[!htb]
    \centering
    \begin{minipage}{0.45\linewidth}
      \centering
      \includegraphics[width=\linewidth]{#1}
      \caption{#2}
    \end{minipage}
    \hspace{0.25cm}
    \begin{minipage}{0.45\linewidth}
      \centering
      \includegraphics[width=\linewidth]{#3}
      \caption{#4}
    \end{minipage}
  \end{figure}
}

\begin{document}

\frame{\titlepage}

\begin{frame}
\frametitle{Indice}
\tableofcontents
\end{frame}

\section{Introduzione}

\begin{frame}
\frametitle{Swarm Robotics}
  \putimage{images/swarm\_robotics.JPG}{0.9}
\end{frame}

\begin{frame}
\frametitle{ARGoS}
\putimage{images/argos3.png}{0.9}
\end{frame}

\begin{frame}
\frametitle{Architettura di ARGoS}
\putimage{images/architecture.png}{0.9}
\end{frame}

\section{La Simulazione}

\begin{frame}
\frametitle{Modello della Simulazione}
\putimage{images/simulation-model.png}{0.9}
\end{frame}

\begin{frame}
\frametitle{Task Allocator}
\putimage{images/task-allocator.png}{0.9}
\end{frame}

\begin{frame}
\frametitle{Task Executor}
\putimage{images/task-executor.png}{0.9}
\end{frame}

\begin{frame}
\frametitle{Task Executor: I Potenziali}
Detta $d$ la distanza tra due corpi soggetti alla forza repulsiva (i droni), $A$ un moltiplicatore specifico di ogni potenziale utilizzato per ottimizzarne l'intensit\`a e $D$ una distanza media che si vuole mantenere tra due droni, si riportano le formule per ricavare le forze di attrazione:

\begin{itemize}
  \item $GP(d) = -A_{GP} \frac {|D - d|} {d}$
  \item $JP(d) = -A_{JP} \frac {D - d} {d^2}$
  \item $LP(d) = -A_{LP} 4 ({\frac {D} {d}}^6 - {\frac {D} {d}}^{12})$
\end{itemize}

Si riportano anche i valori dei coefficienti $A$ che abbiamo utilizzato:

\begin{itemize}
  \item $A_{GP} = 4.0$
  \item $A_{JP} = 16.0$
  \item $A_{LP} = 0.2$
\end{itemize}
\end{frame}

\section{Esempio}

\begin{frame}
\frametitle{Situazione Iniziale}
\putimage{images/esempio/iterazione\_zero\_left.png}{0.9}
\end{frame}

\begin{frame}
\frametitle{Fase di Ascesa}
\putimage{images/esempio/iterazione\_venti\_left.png}{0.9}
\end{frame}

\begin{frame}
\frametitle{Situazione Finale}
\putimage{images/esempio/iterazione\_finale\_top.png}{0.9}
\end{frame}

\section{Esperimenti}

\subsection{Task Executor}

\begin{frame}
\frametitle{I Potenziali a Confronto / 1}
\end{frame}

\begin{frame}
\frametitle{I Potenziali a Confronto / 2}
\end{frame}

\begin{frame}
\frametitle{I Potenziali a Confronto / 3}
\end{frame}

\begin{frame}
\frametitle{Decollo Verticale e Decollo Diretto / 1}
\end{frame}

\begin{frame}
\frametitle{Decollo Verticale e Decollo Diretto / 2}
\end{frame}

\begin{frame}
\frametitle{Decollo Verticale e Decollo Diretto / 3}
\end{frame}

\subsection{Task Allocator}

\begin{frame}
\frametitle{Scelta Iniziale: Random vs Nearest / 1}
\end{frame}

\begin{frame}
\frametitle{Scelta Iniziale: Random vs Nearest / 2}
\end{frame}

\begin{frame}
\frametitle{Fase di Review / 1}
\end{frame}

\begin{frame}
\frametitle{Fase di Review / 2}
\end{frame}

\section{I Limiti}

\begin{frame}
\frametitle{I Limiti}
\centering
\Huge
ARGoS
\end{frame}

\begin{frame}
\frametitle{I Limiti}
\centering
\Huge
Forze di Separazione
\end{frame}

\begin{frame}
\frametitle{I Limiti}
\centering
\Huge
Modellazione delle Collisioni
\end{frame}

\begin{frame}
\frametitle{I Limiti}
\centering
\Huge
Criterio di Arresto
\end{frame}

\begin{frame}
\frametitle{I Limiti}
\centering
\Huge
Generalit\`a
\end{frame}

\section{Conclusioni}

\begin{frame}
\centering
\Huge
Conclusioni
\end{frame}

\begin{frame}
\centering
\Huge
Fine
\end{frame}

\end{document}
